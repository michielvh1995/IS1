% Practicum IS
\documentclass[]{article}
\usepackage{enumerate}
\usepackage{amsfonts}
\usepackage{amssymb}
\usepackage{fullpage}
\usepackage{soul}

\title{Practicum IS 1}
\author{}
\date{}
\begin{document}

  \maketitle
  \subsection*{1}
  \begin{enumerate}[a]
    \item Herschrijven
    \begin{enumerate}[i]
      \item $ \forall x : typeOf(x,vis) \longrightarrow typeOf(vis, zoogdier) $
      \begin{itemize}
        \item $typeOf(x,y) $ geeft aan of $x$ van het type $y$ is (hier: of $vis$ een $zoogdier$ is)
        \item $zoogdier$ is een constante voor het type zoogdier
        \item $vis$ is een constante voor het type vis
      \end{itemize}
      % ii
      \item $ \forall x, \exists y : typeOf(x,vis) \longrightarrow (heeft(x, y) \land typeOf(y,staart)) $
      \begin{itemize}
        \item $heeft(x,y) $ geeft aan of $x$ $y$ heeft
        \item $staart$ is een constante om het type staart aan te geven
      \end{itemize}
      % iii
      \item $\forall x : heeft(x,griep) \longrightarrow heeft(x,koorts) $
      \begin{itemize}
        \item $griep$ is een constante om griep aan te geven
        \item $koorts$ is een constante om koorts aan te geven
      \end{itemize}
      % iv
      \item $ \forall x, \exists y : typeOf(x,docent) \longrightarrow (heeft(x, y) \land typeOf(y,vak)) $
      \begin{itemize}
        \item $docent$ is een constante om het type docent aan te geven
        \item $vak$ is een constante om het type vak aan te geven
      \end{itemize}
      % v
      \item $ \exists x,y : typeOf(x,docent) \land \neg heeft(x,y) \land typeOf(x,vak) $
      % vi
      \item $ Tante(x,y) = typeOf(x,vrouw) \land (neefVan(y,x) \lor nichtVan(y,x)) $
      \begin{itemize}
        \item $vrouw$ is een constante om het type vrouw aan te geven
        \item $neefVan(x,y)$ geeft aan of $x$ de neef van $y$ is
        \item $nichtVan(x,y)$ geeft aan of $x$ de nicht van $y$ is
      \end{itemize}
      % vii
      \item $Deelt(x,y) = \exists z \in \mathbb{N}: z * x = y$
      % viii
      \item $ \forall x, \exists y: typeOf(x,paart) \longrightarrow (\neg heeft(x,y) \land typeOf(y,staart)) $
      \begin{itemize}
        \item $paard$ is een constante om het type paard aan te geven
      \end{itemize}
      % ix : geen paard heeft geen staart -> alle paarden hebben een staart
      \item $ \forall x, \exists y: typeOf(x,paart) \longrightarrow (heeft(x,y) \land typeOf(y,staart)) $
      % x
      \item $ \forall x, \exists y: broerVan(y,x) \longrightarrow \neg lust(x,thee) $
      \begin{itemize}
        \item $broerVan(x,y)$ geeft aan of $x$ de broer is van $y$
        \item $lust(x,y)$ geeft aan of $x$ iets van het type $y$ lust
        \item $thee$ is een constante om thee aan te geven
      \end{itemize}
      % ix
      \item $kwadraatGetal(x) = \exists y \in \mathbb{N}: y*y = x $
    \end{enumerate}

    % einde a
    \item $ ambigue uitspraken $
    \begin{enumerate}[i]
      \item
      \begin{enumerate}[1]
        \item Er is een persoon, en iedereen houdt van hem/haar \\ $\exists x, \forall y: houdtVan(y,x)$
        \item Er is voor iedereen iemand om van te houden \\  $\forall y, \exists x: houdtVan(y,x)$
      \end{enumerate}
      \begin{itemize}
        \item $houdtVan(x,y)$ geeft aan dat $x$ van $y$ houdt
      \end{itemize}
      \item
      \begin{enumerate}[1]
        \item Er is voor iedereen een persoon om van te houden en dit persoon houdt van hem/haar \\  $\exists! z \exists x \forall y: houdtVan(x,z) \longrightarrow houdtVan(y,x)$
        \item Voor iedereen is er een persoon om van te houden en dit persoon houdt ook van hem/haar \\ $\forall y \exists x: houdtVan(x,y) \longrightarrow houdtVan(y,x)$
      \end{enumerate}
      \item
      \begin{enumerate}[1]
        \item Tim tilt de jongen die een stok vast heeft op  \\ $\exists Tim, y, z: heeft(y,z)\land typeOf(z,stok) \longrightarrow tiltOp(Tim,y) $
        \item Tim gebruikt de stok om de jongen op te tillen \\ $\exists Tim, y, z: tiltOpMet(Tim,y,z) $
      \end{enumerate}
      \begin{itemize}
        \item $tiltOp(x,y)$ geeft aan dat $y$ door $x$ wordt op getild
        \item $tiltOpMet(x,y,z)$ geeft aan dat $y$ door $x$ wordt opgetild, met behulp van $z$
        \item $stok$ is een constante die het type stok aan geeft
      \end{itemize}
      \item %iv
      \begin{enumerate}[1]
        \item Er is geen partner-koppel waar tussen het klikt \\ $\forall x, \exists y : partners(x,y) \longrightarrow \neg klikt(x,y)$
        \item Er is een persoon waarmee niet een persoon, dat een partner is van iemand, klikt \\ $\exists z, \forall x, \exists y: partners(x,y) \longrightarrow klikt(z,y) $
      \end{enumerate}
      \begin{itemize}
        \item $partners(x,y)$ geeft aan dat $x$ en $y$ partners zijn
        \item $ klikt(x,y)$ geeft aan of het klikt tussen $x$ en $y$
      \end{itemize}
      \item % v
      \begin{enumerate}[1]
        \item Iedereen kan zowel boeken kopen als boeken lezen bij Bruna of Ako \\ $\forall x \exists y: kanBoekLezenBij(x,y) \land kanBoekKopenBij(x,y) \land (typeOf(y,Ako) \lor typeOf(y,Bruna)) $
        \item Iedereen kan boeken kopen bij Bruna of boeken lezen bij de Ako. \\ $\forall x \exists y: (kanBoekKopenBij(x,y) \land typeOf(y,Bruna)) \lor ((kanBoekLezenBij(x,y) \land  typeOf(y,Ako)) )  $
      \end{enumerate}
      \begin{itemize}
        \item $kanBoekLezenBij(x,y)$ geeft aan dat $x$ op locatie $y$ een boek kan lezen
        \item $kanBoekKopenBij(x,y)$ geeft aan dat $x$ op locatie $y$ een boek kan kopen
        \item $Ako$ is een constantie die een Ako winkel aangeeft
        \item $Bruna$ is een constantie die een Bruna winkel aangeeft
      \end{itemize}
      \item % vi
      \begin{enumerate}[1]
        \item Als de zon onder is de maan of de zon onzichtbaar, of beide \\ $\exists zon, maan : (onder(zon) \longrightarrow (onzichtbaar(maan) \lor onzichtbaar(zon)))$
        \item De zon kan onzichtbaar zijn, maar als hij onder is is de maan onzichtbaar \\ $\exists zon, maan : (onder(zon) \longrightarrow onzichtbaar(maan))\lor onzichtbaar(zon)$
      \end{enumerate}
      \begin{itemize}
        \item $onder(x)$ geeft aan of $x$ onder de horizon verdwenen is
        \item $onzichtbaar(x)$ geeft aan of $x$ niet zichtbaar is
      \end{itemize}
    \end{enumerate} % einde b
  \end{enumerate}

  \subsection*{2}
  \begin{enumerate}[a]
    \item
    \begin{enumerate}[i]
      \item
      \begin{enumerate}[1]
        %\item $\forall y, x : kniptHaar(y,y) \longrightarrow \neg (kniptHaar(x,y) \land typeOf(x,kapper)) $
        \item $  \forall x, y: \lnot (typeOf(x, kapper) \land typeOf(y,persoon) \land knipt(x,y) \land knipt(y,y)) $
        \begin{itemize}
          \item $knipt(x,y)$ geeft aan dat $x$ de haren van $y$ knipt.
          \item $kapper$ is een constante om het type kapper aan te geven.
          \item $persoon$ is een constante om het type persoon aan te geven.
        \end{itemize}
        %\item $\forall x,y : \neg kniptHaar(y,y) \longrightarrow (kniptHaar(x,y) \land typeOf(x,kapper)) $
        \item $ \forall x, y: typeOf(x, kapper) \land typeOf(y, persoon) \land \lnot knipt(y, y) \longrightarrow knipt(x, y) $
        %\item $\forall x: typeOf(x,kapper) \longrightarrow typeOf(x,persoon)$
        \item $ \forall x: typeOf(x, kapper) \longrightarrow typeOf(x, persoon)$
      \end{enumerate}
      % ii
      \item $\forall x: \neg typeOf(x,kapper)$
      % iii
      \item Formules (1), (2) en (3)
      % iv
      \item We schrijven eerst alle formules om naar CNF equivalenten. Vervolgens plakken we al die formules aan elkaar met $\land$ -tekens en
      gaan we op zoek naar literals die tegenstrijdig zijn. Op deze manier kan je concluderen dat de waarde van de literals niet uit maakt en ze verwijderen
      uit de gecombineerde formule. Dit proces gaat door tot dat we de formule die we proberen af te leiden over houden.
      % v
      \item
      \begin{enumerate}[1]
        %\item $ (\neg kniptHaar(y,y) \lor \lnot kniptHaar(x,y)) \land (\lneg kniptHaar(y,y) \lor typeOf(x,kapper)) $
        \item $\lnot typeOf(x, kapper) \lor \lnot typeOf(y, persoon) \lor \lnot knipt(x, y) \lor \lnot knipt(y, y)$
        %\item $ (kniptHaar(y,y) \lor kniptHaar(x, y)) \land (kniptHaar(y,y) \lor typeOf(x,kapper))$
        \item $ \lnot typeOf(x, kapper) \lor \lnot typeOf(y, persoon) \lor knipt(y, y) \lor knipt(x, y) $
        \item $ \lnot typeOf(x,kapper) \lor typeOf(x,persoon) $
      \end{enumerate}
      % vi
      \item Zoals gezegd zetten we alles onder elkaar:
      \begin{enumerate}[1]
        \item $\lnot typeOf(x, kapper) \lor \lnot typeOf(y, persoon) \lor \lnot knipt(x, y) \lor \lnot knipt(y, y)$
        \item $ \lnot typeOf(x, kapper) \lor \lnot typeOf(y, persoon) \lor knipt(y, y) \lor knipt(x, y) $
        \item $ \lnot typeOf(x,kapper) \lor typeOf(x,persoon) $
      \end{enumerate}
      \item Vervolgens gaan we overbodigen, ofwel tegenspraak wegstrepen. We beginnen bij knipt:
      \begin{enumerate}[1]
        \item $\lnot typeOf(x, kapper) \lor \lnot typeOf(y, persoon) \lor$\st{$ \lnot knipt(x, y) \lor \lnot knipt(y, y)$}
        \item $ \lnot typeOf(x, kapper) \lor \lnot typeOf(y, persoon) \lor$\st{$knipt(y, y) \lor knipt(x, y)$}
        \item $ \lnot typeOf(x,kapper) \lor typeOf(x,persoon) $
      \end{enumerate}
      \item Dit kunnen we nog een keer doen, deze keer kijken we naar typeOf met als 2e argument "persoon" omdat we uiteindelijk kapper willen overhouden.
      \begin{enumerate}[1]
        \item $ \lnot typeOf(x, kapper) \lor $\st{$\lnot typeOf(y, persoon)$}$ \lor$\st{$ \lnot knipt(x, y) \lor \lnot knipt(y, y)$}
        \item $ \lnot typeOf(x, kapper) \lor $\st{$\lnot typeOf(y, persoon)$}$ \lor$\st{$knipt(y, y) \lor knipt(x, y)$}
        \item $ \lnot typeOf(x,kapper) $\st{$\lor typeOf(x,persoon) $}
      \end{enumerate}
      \item We zien dat we nu alleen nog de af te leiden formule over hebben (in CNF), dus de resolutie heeft gewerkt.
    \end{enumerate}
    \item 2b

  \end{enumerate}

  \subsection*{3}
  \begin{enumerate}[a]
    \item Sudoku Variabelen:
    \begin{itemize}
      \item $ [r_1 .. r_9] $ voor alle rijen, deze geven aan of ze correct ingevuld zijn
      \item $ [k_1 .. k_9] $ voor alle kolommen, deze geven aan of ze correct ingevuld zijn
      \item $ [b_1 .. b_9] $ voor alle blokken, deze geven aan of ze correct ingevuld zijn
      \item $ [1Op(i,j) .. 9Op(i,j)] $ deze variabelen geven de waarde aan van vakje $i,j$. Hiervan kan er slechts een per $i,j$ combinatie $true$ zijn
    \end{itemize}
    \item In totaal zijn er 108 variabelen $true$:
    \begin{itemize}
      \item $ \forall i \in [1..9] : r_i $, $b_i$ en $k_i$ zijn $true$.
      \item Van $ [1Op(i,j) .. 9Op(i,j)] $ zijn slechts 81 variabelen $true$.
      \item Ook geldt dat $\forall i,n \in [1..9]$ alle $n$ een keer voorkomt.
      \item Tot slot geldt dat $\forall j,n \in [1..9]$ alle $n$ een keer voorkomt.
    \end{itemize}
    \item $\forall i,j \in [1..9]: 1Op(i,j) \lor .. \lor 9Op(i,j) $
    \item Voor elk blok: $  $
  \end{enumerate}






\end{document}
